\documentclass[a4paper,11pt]{article}

% packages and settings [[[

% input	[[[
\usepackage[utf8]{inputenc}
\usepackage{ngerman}
% ]]]
% math/symbols [[[

\usepackage{latexsym}
\usepackage{amsfonts}	
\usepackage{amsmath}
\usepackage{amssymb}

% \usepackage{wasysym}	\Smile
% ]]]
% tables/graphics/algorithms/listings [[[

\usepackage{booktabs}
\usepackage{multirow}
\usepackage[dvips]{graphicx}
\usepackage{listings}
\usepackage{algorithmic}
\usepackage[boxed]{algorithm}
\usepackage{subfigure}
\usepackage[margin=10pt,font=small,labelfont=bf]{caption}		
	% \caption zb. in \figure wird \small

% settings for listing environment
\lstset{language=C++,
		basicstyle=\small,
		frame=single,
        breaklines=true,
        breakatwhitespace=true,
		numbers=left,
		numberstyle=\tiny,
		xleftmargin=5mm,
		tabsize=4
}

% nicer comment formatting in algorithmic env.
\renewcommand{\algorithmiccomment}[1]{\hspace{\stretch{1}}// #1$\quad$}
\floatname{algorithm}{Algorithmus}

% drawin trees
% \usepackage{pstricks,pst-node,pst-tree}

% ]]]
% paragraph formatting [[[
% \usepackage{a4wide}

\usepackage{parskip}
\parindent0pt
\setlength{\parskip}{1ex plus 0.5ex minus 0.2 ex}

% ]]]

% ]]]
% my commands [[[

% my math commands [[[

\newcommand{\R}{\mathbb{R}}
\newcommand{\N}{\mathbb{N}}
\newcommand{\mts}{\;|\;}
\newcommand{\eqs}{\;=\;}
\newcommand{\D}{\mathrm{d}}
\newcommand{\Z}{\mathbb{Z}}
\renewcommand{\P}{\mathbb{P}}
\newcommand{\area}{\mathrm{area}}
\newcommand{\matrixtwo}[2]{\left(\begin{array}{cc}#1\\#2\end{array}\right)}
\newcommand{\matrixthree}[3]{\left(\begin{array}{ccc}#1\\#2\\#3\end{array}\right)}
\newcommand{\matrixfour}[4]{\left(\begin{array}{cccc}#1\\#2\\#3\\#4\end{array}\right)}
\newcommand{\vect}[3]{\left(\begin{array}{c}#1\\#2\\#3\end{array}\right)}
\newcommand{\vectfour}[4]{\left(\begin{array}{c}#1\\#2\\#3\\#4\end{array}\right)}
\newcommand{\norm}{\mathrm{norm}}
\newcommand{\scalPr}[2]{\left\langle #1|#2 \right\rangle}

% ]]]
% definitions [[[

\newcounter{currdefinition}
\newcommand{\defnr}[1]{\arabic{#1}}
\newcommand{\defref}[1]{Definition \defnr{#1}}
\newcommand{\definition}[2]
	{	\addtocounter{currdefinition}{1}%
		\newcounter{#1}%
		\setcounter{#1}{\thecurrdefinition}%
		\label{#1}%
		\paragraph{\textbf{Definition \thecurrdefinition:}}\ \\#2}

% ]]]
% misc commands [[[

\newcommand{\ra}{$\rightarrow\ $}
\newcommand{\wichtig}[1]{\framebox{\begin{minipage}[h]{\textwidth}#1\end{minipage}} }
\newcommand{\nachvollz}[1]{\paragraph{\textbf{Nachvollziehen:}}\ \\#1}
\newcommand{\tableitemization}{\hspace{10pt}$\bullet$&}
\newcommand{\HELP}[1]{\ \\\ \\\Huge Hilfe! \normalsize\\$\rightarrow$\ #1\ \\\ \\}

\newcommand{\inlinecode}[1]{{\tt #1}}
\newcommand{\inlineshell}[1]{{\tt #1}}

% ]]]
% levels [[[

\newcommand{\levelA}[1]{\section{#1}}
\newcommand{\levelB}[1]{\subsection{#1}}
\newcommand{\levelC}[1]{\subsubsection{#1}}
\newcommand{\levelD}[1]{\paragraph{#1:}\ \\}
\newcommand{\levelDn}[1]{\paragraph{#1:}}    % level D ohne "\ \\" --> bei paragraph oder itemization direkt danach sonst zu großer zwischenraum
\newcommand{\levelE}[1]{\paragraph{#1:}\ \\}
\newcommand{\levelEn}[1]{\paragraph{#1:}}    % level D ohne "\ \\" --> bei paragraph oder itemization direkt danach sonst zu großer zwischenraum

% ]]]

% ]]]
\usepackage[numbers]{natbib}


\title{Anmerkungen zur Stackless BVHs}
\author{Kai}
\date{Last built: \today}

% \setlength{\unitlength}{1mm}

\begin{document}

\maketitle\thispagestyle{empty}
% \tableofcontents

\section{Intro}
Die normale Stackless-Sache siehe \citet{DAENEN} zeigt entweder richtungsabhängige Performance oder ist langsamer als sie sein könnte.

Der erste Fall tritt auf, wenn beim Erstellen der Struktur die Knoten die jeweils den entsprechenden ``negativen'' Teilbaum darstellen einheitlich als linker oder rechter Teilbaum einsortiert werden.
Dadurch ergbit sich, dass die Struktur von einer Richtung aus passend aufgebaut ist (d.h. die Knoten näher am Strahl werden auch zuerst traversiert), von der invertierten Richtung aus ist diese Struktur sehr ungünstig.

Um das zu vermeiden sortieren \citet{DAENEN} die Knoten zufällig um, so dass der Baum kein direktes Bias mehr har.
Das führt aber natürlich dazu, dass nicht nur der Worst-Case verschwindet, sondern auch der Best-Case.

\subsection{Vergleich BVH/SBVH}

\subsection{Außen vs.\ Innen}
Die für die Richtungsabhängigkeit aufgeführte Argumentation ist sehr anschaulich, wenn man sich die Strahlen von außerhalb der Szene stammend vonstellt.
Wie verhält es sich, wenn die Strahlen z.B. vom Szenenmittelpunkt (z.B. in Sponza) geschossen werden?

\emph{These.} Es wird sehr schnell erkannt dass Teilbäume nicht traversiert werden müssen weil sie hinter der Kamera liegen.

Das muss aber noch überprüft werden.


\subsection{Relation zu naivem BVH-Traversal}
Beim Traversieren von BVHs werden üblicherweise die Boxen der Kinder eines Knotens mit dem Strahl geschnitten.
Aus dieser Berechnung geht auch die Distanz der jeweiligen Konten entlang des Strahls hervor, und gemäß dieser werden die Kindknoten korrekt sortiert auf den Stack abgelegt (so, dass der näheste zuerst traversiert wird).

Eine naivere Traversionsmethode ist den betrachteten Knoten selbst zu schneiden und die Kindknoten in beliebiger Reihenfolge auf den Stack zu legen.
In dem Fall ergibt sich eine ähnliche Situation wie bei der SBVH: werden die Knoten nach einem bestimmten Musten auf den Stack abgelegt schlägt sich das Richtungsabhängig nieder, wird randomisiert (was an der Stelle (aka inner loop) per se keine gute Idee ist) erhält man nur Average-Case Performance.


\subsection{Traversion ohne Skip-Pointer}
\begin{tabular}{ll}
$L(k,d)$	&	left child of $k$ (when the ray direction is in octant $d$) \\
$R(k,d)$	&	right child of $k$ (ditto) \\
$P(k)$		&	parent of $k$\\
$N(k,d)$	&	$\left\{\begin{array}{lr} L(k,d)	&	\text{$k$ inner node}\\
                                          S(k,d)	&	\text{$k$ leaf node}\\
										  \end{array}\right.$	\\
$S(k,d)$	&	$\left\{\begin{array}{lr} S(P(k),d)	&	R(P(k),d) = k\\
                                          R(P(k),d)	&	\text{otherwise}\\
										  \end{array}\right.$
\end{tabular}

\bibliographystyle{plainnat}
\bibliography{/home/kai/orga/bib}

\end{document}

% vim: set foldmethod=marker foldmarker=[[[,]]]: 


